% !TeX spellcheck = en_US
\documentclass[10pt,a4paper]{article}
\usepackage[latin1]{inputenc}
\usepackage[T1]{fontenc}
\usepackage{amsmath}
\usepackage{amsfonts}
\usepackage{amssymb}
\usepackage{graphicx}
\author{OcamlPro - Fabrice Le Fessant et Thomas Sibut-Pinote}
\title{Groth16 zkSNARK Proof Verification : Three Use Cases}
\begin{document}
\maketitle
In this submission, we propose three different use cases for Groth16
zkSNARK proofs on the FreeTON blockchain.

\section{Technical contribution to the usability: `ft`}

(justifi� parce que les jur�s peuvent se servir de `ft` pour tester
notre code)

\section{A toy example: Sudoku}
\label{section_sudoku}

Let's start with our first toy use case. Suppose you want to set up a Sudoku problem for your students to solve, so that upon completion they receive some token. The issue is that once any student has solved it, the solution sits on the blockchain for all to see. All the other students can cheat by copying it, instead of doing the work by themselves, which defeats the whole purpose.

\section{A better example: Project Euler}
\label{section_euler}

The previous example is somewhat limited by the fact that sudokus are easily solvable problems, either by hand or with a computer; they are not a realistic use case of zk-snarks. Let's keep the idea of a decentralized classroom. the famous website Project Euler\footnote{{https://projecteuler.net/}} offers math and programming problems of various difficulty to its users. It uses captchas to prevent brute force attempts at guessing the answer to a problem. The answers to problems are stored statically on its servers and the answers are checked against them. Can we make Project Euler decentralized? Using the idea from Section \ref{section_sudoku}, we can prevent users from stealing others' solutions \textbf{and} from brute-forcing them, due to the cost of generating proofs.


\section{A useful example: SetCodeMultisigWallet with a pin code}
\label{section_pincode}



\end{document}